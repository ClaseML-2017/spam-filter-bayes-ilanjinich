\chapter{Introducción}
\begin{flushright}
\textit{``Matchmaker, matchmaker,
Make me a match, Find me a find,
Catch me a catch''}
\end{flushright}
\section{El problema de admisión a universidades}
Descripción del problema



\begin{dfn}{(Gale Shapley 1962)}
\label{Estable}
Decimos que una asignación de solicitantes a universidades es inestable si existen dos solicitantes $\alpha$ y $\beta$ asignados a  universidades $A$ y $B$ respectivamente, con la propiedad de que $\alpha$ prefiere estar en $B$ que en $A$ y $B$ prefiere tener a $\alpha$ que a $B$. Es decir, existen una universidad y un solicitante que se prefieren entre ellos a sus respectivas asignaciones. Alternativamente una asignación de solicitantes a universidades es estable si no es inestable.
\end{dfn}

\begin{dfn}{(Gale Shapley 1962)}
\label{optima}
Una asignación es considera optima si cada solicitante esta mejor o igual respecto a sus preferencias que en cualquier otra asignación estable. 
\end{dfn}



%DEscripcion
%Definiciones
%Chane esto jala mejor como capitulo y no seccion
%COmentario que explique como abajo es caso particular

\section{El problema del matrimonio estable}
\begin{flushright}
\textit{``Even the worst husband, God forbid, is better than no husband, God forbid.''}
\end{flushright}
%Definicion del problema
%Definiciones alternativas
\begin{dfn}
Definimos la matriz de preferencias para un problema con $n$ y con $m$ mujeres como una matriz con $n$ filas y $m$ columnas donde donde la primera entrada de la celda $(i,j)$ represente el orden de preferencia que le asigna el hombre $i$ a la mujer $j$ y análogamente la segunda entrada representa el orden de preferencia que le da la mujer $j$ al hombre $i$.
\end{dfn}
\begin{eje}
\label{ejemplo matrimonio 1}
Supongamos que contamos con tres hombres $\alpha$, $\beta$ y $\gamma$ y con tres mujeres $A$, $B$ y $C$ y la matriz de preferencias
$$\begin{pmatrix}
& A & B & C \\
\alpha & 1,3 & 2,2 & 3,1 \\
\beta & 3,1 & 1,3 & 2,2 \\
\gamma & 2,2 & 3,1 & 1,3 
\end{pmatrix}$$
aquí por ejemplo el orden preferencias de $\alpha$ es $(A,B,C)$ y el orden de preferencias de $C$ es $(\gamma, \beta, \alpha)$.
\end{eje}

\begin{dfn}
Análogo a la definición \ref{Estable},
decimos que un conjunto de matrimonios es inestable si existe un hombre y una mujer que se prefieren entre si a sus respectivas parejas. Alternativamente un conjunto de matrimonios es estable si no es inestable. 
\end{dfn}

%Ejemplos
\begin{eje}
Retomando el ejemplo \ref{ejemplo matrimonio 1}. \\
El conjunto de parejas $(\alpha, A), \; (\beta, B)\; y\; (\gamma, C)$ es estable porque a pesar de las preferencias de las mujeres cada hombre esta con su primera opción, es decir, los hombres prefieren a su pareja a cualquier otra mujer.\\ 

\end{eje}
\begin{eje}
Retomando el ejemplo \ref{ejemplo matrimonio 1}. \\
El conjunto de parejas $(\alpha, A)$, $(\gamma, B)$ y $(\beta, C)$ es inestable porque $\gamma$ prefiere a $A$ que a su pareja actual ($B$) y A prefiere a $\gamma$ que a su pareja actual $(\alpha)$.
\end{eje}
\subsection{Algoritmo de Gale Shapley}
\begin{eje}
Ejemplo de 4x4
\end{eje}

\begin{teo}[Teorema de Gale Shaley]
\label{teorema de Gale Shapley}
El algoritmo de Gale Shapley termina en una emparejamiento estable.
\end{teo}
\begin{proof}

\end{proof}

\begin{lem}
\label{lema 1}
Bajo el emparejamiento obtenido por Gale Shapley, solo un hombre puede terminar con la última mujer de su lista como pareja. 
\end{lem}

\begin{proof}
Supongamos que en el emparejamiento de Gale Shapley $m$ ($m\geq2$) hombres terminan con la última mujer de su lista como pareja, eso significa que cada uno de esos $m$ hombres invito a salir a todas las mujeres. Entonces cada mujer fue invita a salir por lo menos $m$ veces, lo cual es una contradicción porque el algoritmo acaba cuando invitan a salir a la última mujer y a esta solo la invitan a salir una vez. Por lo tanto, solo un hombre puede terminar con la última mujer de su lista como pareja. 
\end{proof}

\begin{cor}
Si en un emparejamiento estable por lo menos dos hombres están emparejados con la última mujer de sus respectivas listas, entonces existe dos o más emparejamientos estables en el problema.
\end{cor}

\begin{proof}
Llamemos $m$ al emparejamiento estable donde por lo menos dos hombres están emparejados con la última mujer de sus respectivas listas y llamemos $m'$ al emparejamiento de Gale Shapley.
Por el teorema \ref{teorema de Gale Shapley} sabemos que el algoritmo de Gale Shapley siempre acaba en un emparejamiento estable y además en ese emparejamiento estable solo un hombre puede terminar con la última mujer de su lista como pareja por el lema \ref{lema 1}.
Por lo tanto $m$ y $m'$ son diferentes y como ambos son emparejamientos estables entonces el número de emparejamientos estables en el problema es mayor o igual a dos. 
\end{proof}

\begin{cor}
El número máximo de propuestas en el algoritmo es $n^2-n+1$
\end{cor}

\begin{proof}
Por el lema \ref{lema 1} sabemos que a lo más un hombre acaba emparejado con la última mujer de su lista, por lo tanto el peor emparejamiento posible para el algoritmo es uno donde $n-1$ hombres terminan con la penúltima mujer de sus respectivas listas y un hombre termina con la última. Para llegar a esta situación de acuerdo al algoritmo, los $n-1$ hombres deben de realizar $n-1$ propuestas cada uno y el otro hombre debe de realizar $n$ propuestas. Esto es $(n-1)(n-1)+n$ propuestas que es igual a $n^2-n+1$ propuestas.
\end{proof}

\begin{obs}
La complejidad del algoritmo de Gale Shapley es del orden de $n^2$.
\end{obs}

\begin{eje}
Aquí va el ejemplo de 2x2 del lema
Considere la matriz de preferencias 
$$\begin{pmatrix}
&A&B\\
\alpha &1,1& 2,1 \\
\beta & 1,2 & 2,2
\end{pmatrix}$$
\end{eje}

\begin{eje}
Aquí va el ejemplo de 3x3 del lema
\\ Considere la matriz de preferencias $$\begin{pmatrix}
& A & B & C \\
\alpha & 1,2 & 3,2 & 2,1 \\
\beta & 2,3 & 3,1 & 1,2 \\
\gamma & 2,1 & 3,3 & 1,3 
\end{pmatrix}$$
\end{eje}


